% Options for packages loaded elsewhere
\PassOptionsToPackage{unicode}{hyperref}
\PassOptionsToPackage{hyphens}{url}
%
\documentclass[
]{article}
\usepackage{amsmath,amssymb}
\usepackage{iftex}
\ifPDFTeX
  \usepackage[T1]{fontenc}
  \usepackage[utf8]{inputenc}
  \usepackage{textcomp} % provide euro and other symbols
\else % if luatex or xetex
  \usepackage{unicode-math} % this also loads fontspec
  \defaultfontfeatures{Scale=MatchLowercase}
  \defaultfontfeatures[\rmfamily]{Ligatures=TeX,Scale=1}
\fi
\usepackage{lmodern}
\ifPDFTeX\else
  % xetex/luatex font selection
\fi
% Use upquote if available, for straight quotes in verbatim environments
\IfFileExists{upquote.sty}{\usepackage{upquote}}{}
\IfFileExists{microtype.sty}{% use microtype if available
  \usepackage[]{microtype}
  \UseMicrotypeSet[protrusion]{basicmath} % disable protrusion for tt fonts
}{}
\makeatletter
\@ifundefined{KOMAClassName}{% if non-KOMA class
  \IfFileExists{parskip.sty}{%
    \usepackage{parskip}
  }{% else
    \setlength{\parindent}{0pt}
    \setlength{\parskip}{6pt plus 2pt minus 1pt}}
}{% if KOMA class
  \KOMAoptions{parskip=half}}
\makeatother
\usepackage{xcolor}
\usepackage[margin=1in]{geometry}
\usepackage{color}
\usepackage{fancyvrb}
\newcommand{\VerbBar}{|}
\newcommand{\VERB}{\Verb[commandchars=\\\{\}]}
\DefineVerbatimEnvironment{Highlighting}{Verbatim}{commandchars=\\\{\}}
% Add ',fontsize=\small' for more characters per line
\usepackage{framed}
\definecolor{shadecolor}{RGB}{248,248,248}
\newenvironment{Shaded}{\begin{snugshade}}{\end{snugshade}}
\newcommand{\AlertTok}[1]{\textcolor[rgb]{0.94,0.16,0.16}{#1}}
\newcommand{\AnnotationTok}[1]{\textcolor[rgb]{0.56,0.35,0.01}{\textbf{\textit{#1}}}}
\newcommand{\AttributeTok}[1]{\textcolor[rgb]{0.13,0.29,0.53}{#1}}
\newcommand{\BaseNTok}[1]{\textcolor[rgb]{0.00,0.00,0.81}{#1}}
\newcommand{\BuiltInTok}[1]{#1}
\newcommand{\CharTok}[1]{\textcolor[rgb]{0.31,0.60,0.02}{#1}}
\newcommand{\CommentTok}[1]{\textcolor[rgb]{0.56,0.35,0.01}{\textit{#1}}}
\newcommand{\CommentVarTok}[1]{\textcolor[rgb]{0.56,0.35,0.01}{\textbf{\textit{#1}}}}
\newcommand{\ConstantTok}[1]{\textcolor[rgb]{0.56,0.35,0.01}{#1}}
\newcommand{\ControlFlowTok}[1]{\textcolor[rgb]{0.13,0.29,0.53}{\textbf{#1}}}
\newcommand{\DataTypeTok}[1]{\textcolor[rgb]{0.13,0.29,0.53}{#1}}
\newcommand{\DecValTok}[1]{\textcolor[rgb]{0.00,0.00,0.81}{#1}}
\newcommand{\DocumentationTok}[1]{\textcolor[rgb]{0.56,0.35,0.01}{\textbf{\textit{#1}}}}
\newcommand{\ErrorTok}[1]{\textcolor[rgb]{0.64,0.00,0.00}{\textbf{#1}}}
\newcommand{\ExtensionTok}[1]{#1}
\newcommand{\FloatTok}[1]{\textcolor[rgb]{0.00,0.00,0.81}{#1}}
\newcommand{\FunctionTok}[1]{\textcolor[rgb]{0.13,0.29,0.53}{\textbf{#1}}}
\newcommand{\ImportTok}[1]{#1}
\newcommand{\InformationTok}[1]{\textcolor[rgb]{0.56,0.35,0.01}{\textbf{\textit{#1}}}}
\newcommand{\KeywordTok}[1]{\textcolor[rgb]{0.13,0.29,0.53}{\textbf{#1}}}
\newcommand{\NormalTok}[1]{#1}
\newcommand{\OperatorTok}[1]{\textcolor[rgb]{0.81,0.36,0.00}{\textbf{#1}}}
\newcommand{\OtherTok}[1]{\textcolor[rgb]{0.56,0.35,0.01}{#1}}
\newcommand{\PreprocessorTok}[1]{\textcolor[rgb]{0.56,0.35,0.01}{\textit{#1}}}
\newcommand{\RegionMarkerTok}[1]{#1}
\newcommand{\SpecialCharTok}[1]{\textcolor[rgb]{0.81,0.36,0.00}{\textbf{#1}}}
\newcommand{\SpecialStringTok}[1]{\textcolor[rgb]{0.31,0.60,0.02}{#1}}
\newcommand{\StringTok}[1]{\textcolor[rgb]{0.31,0.60,0.02}{#1}}
\newcommand{\VariableTok}[1]{\textcolor[rgb]{0.00,0.00,0.00}{#1}}
\newcommand{\VerbatimStringTok}[1]{\textcolor[rgb]{0.31,0.60,0.02}{#1}}
\newcommand{\WarningTok}[1]{\textcolor[rgb]{0.56,0.35,0.01}{\textbf{\textit{#1}}}}
\usepackage{graphicx}
\makeatletter
\def\maxwidth{\ifdim\Gin@nat@width>\linewidth\linewidth\else\Gin@nat@width\fi}
\def\maxheight{\ifdim\Gin@nat@height>\textheight\textheight\else\Gin@nat@height\fi}
\makeatother
% Scale images if necessary, so that they will not overflow the page
% margins by default, and it is still possible to overwrite the defaults
% using explicit options in \includegraphics[width, height, ...]{}
\setkeys{Gin}{width=\maxwidth,height=\maxheight,keepaspectratio}
% Set default figure placement to htbp
\makeatletter
\def\fps@figure{htbp}
\makeatother
\usepackage{soul}
\setlength{\emergencystretch}{3em} % prevent overfull lines
\providecommand{\tightlist}{%
  \setlength{\itemsep}{0pt}\setlength{\parskip}{0pt}}
\setcounter{secnumdepth}{-\maxdimen} % remove section numbering
\ifLuaTeX
  \usepackage{selnolig}  % disable illegal ligatures
\fi
\IfFileExists{bookmark.sty}{\usepackage{bookmark}}{\usepackage{hyperref}}
\IfFileExists{xurl.sty}{\usepackage{xurl}}{} % add URL line breaks if available
\urlstyle{same}
\hypersetup{
  pdftitle={E-commerce Exploratory Analytics \& Dashboard},
  pdfauthor={rs2904},
  hidelinks,
  pdfcreator={LaTeX via pandoc}}

\title{E-commerce Exploratory Analytics \& Dashboard}
\author{rs2904}
\date{2023-11-28}

\begin{document}
\maketitle

{
\setcounter{tocdepth}{2}
\tableofcontents
}
\hypertarget{assignment-overview}{%
\section{Assignment Overview}\label{assignment-overview}}

\hypertarget{background}{%
\subsection{Background}\label{background}}

With the rise of e-commerce, understanding customer behavior and
purchasing patterns is vital for online businesses. Utilizing
transactional data can provide actionable insights, helping to optimize
sales strategies, improve customer experiences, and maximize revenue.
Your primary objective is to assist an online retailer by analyzing
their transaction data to uncover sales trends, customer segmentation,
and potential areas of improvement.

\hypertarget{data-insights}{%
\subsection{Data Insights}\label{data-insights}}

The data set captures transactions from December 2010 to December 2011
from a UK-based online retailer. Fields include:

\begin{itemize}
\tightlist
\item
  \texttt{InvoiceNo} - Invoice number - a 6-digit integral number
  uniquely assigned to each transaction. If this code starts with letter
  `c', it indicates a cancellation.
\item
  \texttt{StockCode} - a 5-digit integral number uniquely assigned to
  each distinct product
\item
  \texttt{Description} - product name
\item
  \texttt{Quantity} - the quantities of each product (item) per
  transaction
\item
  \texttt{InvoiceDate} - the day and time when each transaction was
  generated
\item
  \texttt{UnitPrice} - product price per unit
\item
  \texttt{CustomerID} - a 5-digit integral number uniquely assigned to
  each customer
\item
  \texttt{Country} - the name of the country where each customer resides
\end{itemize}

For more details about the data set, refer to the
\href{https://archive.ics.uci.edu/ml/datasets/Online+Retail}{UCI Machine
Learning Repository}.

\hypertarget{part-1-data-exploration-with-r-60-of-total-marks}{%
\subsection{Part 1: Data Exploration with R (60\% of total
marks)}\label{part-1-data-exploration-with-r-60-of-total-marks}}

\hypertarget{task}{%
\subsubsection{Task}\label{task}}

Based on the data set, determine:

\begin{enumerate}
\def\labelenumi{\arabic{enumi}.}
\tightlist
\item
  What are the most popular products and how do their sales vary over
  time?
\item
  Can we segment customers based on their purchasing behavior?
\item
  Are there specific countries that contribute more to sales or have
  unique buying patterns?
\item
  (Bonus) How do cancellations impact overall sales trends?
\end{enumerate}

\textbf{Reflection}: Beyond the questions above, take a moment to
examine the data set critically.

\begin{itemize}
\tightlist
\item
  Are there any other interesting patterns or anomalies you notice?
\item
  What additional questions or analyses come to mind that could be
  valuable for the retailer?
\item
  How might external factors, not present in the data set, influence the
  trends you've observed?
\end{itemize}

Document your process, findings, and reflections in the current \emph{R
Notebook} - just use the space below this assignment instructions. Be
compact in your descriptive and reflective answers!

\hypertarget{submission-requirements}{%
\subsubsection{Submission Requirements}\label{submission-requirements}}

\begin{itemize}
\tightlist
\item
  The RMD file containing all answers and relevant code.
\item
  An HTML version of the notebook, knitted from the RMD.
\end{itemize}

\textbf{Note}: Ensure your report is: - Structured for easy navigation.
- Includes clear visualizations with apt labels and annotations. -
Provides short, compact answers to the exploratory questions, also
documenting key steps. - Contains efficient and easily executable R
code, preferably using \texttt{\%\textgreater{}\%} pipes for streamlined
data manipulation.

\hypertarget{part-2-tableau-dashboard-40-of-total-marks}{%
\subsection{Part 2: Tableau Dashboard (40\% of total
marks)}\label{part-2-tableau-dashboard-40-of-total-marks}}

\hypertarget{task-1}{%
\subsubsection{Task}\label{task-1}}

Using insights from your R analysis, design a user-friendly, interactive
dashboard in Tableau Desktop. Focus on showcasing significant findings
related to sales trends and customer behaviors. Your target audience is
the CEO of the online retail store - a non-tech-savvy individual keen on
understanding broad trends, customer insights, and avenues for growth.

\begin{enumerate}
\def\labelenumi{\arabic{enumi}.}
\tightlist
\item
  \textbf{Storyboarding}: Prior to Tableau, sketch a dashboard
  storyboard, snapshot or photograph it for the submission.
\item
  \textbf{Dashboard Creation in Tableau}: Transform your R insights into
  a user-centric Tableau dashboard. Highlight crucial trends and
  customer behaviors.
\end{enumerate}

\hypertarget{guidelines}{%
\subsubsection{Guidelines}\label{guidelines}}

\begin{itemize}
\tightlist
\item
  Prioritize clarity and ease of navigation.
\item
  Utilize filters and design elements for an intuitive user experience.
\item
  Add descriptions and headers to guide users through your dashboard.
\end{itemize}

\hypertarget{submission-requirement}{%
\subsubsection{Submission Requirement}\label{submission-requirement}}

\begin{itemize}
\tightlist
\item
  Submit the image of your storyboard sketch;
\item
  Submit the Tableau Packaged Workbook (TWBX) file, including all
  worksheets and the final dashboard.
\end{itemize}

\textbf{Caution}: Always export as a \emph{Tableau Packaged Workbook
(TWBX)}. Exporting as a \emph{Tableau Workbook (TWB)} might omit
important data operations, making your dashboard potentially unreadable.

\begin{center}\rule{0.5\linewidth}{0.5pt}\end{center}

\hypertarget{commence-your-assignment-here}{%
\section{Commence Your Assignment
Here}\label{commence-your-assignment-here}}

\hypertarget{data-import-preliminaries}{%
\subsection{Data Import \&
Preliminaries}\label{data-import-preliminaries}}

Bringing external datasets into R for analysis is known as \textbf{data
import} in the language. R has a number of functions and packages for
importing data from different sources.

The below code snippet will help to import the data set ``Online
Retail.csv'' in R for further analysis.

\begin{Shaded}
\begin{Highlighting}[]
\NormalTok{cw\_data }\OtherTok{\textless{}{-}} \FunctionTok{read.csv}\NormalTok{(}\StringTok{"Online Retail.csv"}\NormalTok{) }\CommentTok{\#the data set is assigned to the variable cw\_data}
\end{Highlighting}
\end{Shaded}

After the import, the below code will help to start the analysis of the
data by giving the head and summary of the data set.

\textbf{head():} For checking the first 5 entries of the data set and
along with the column names to understand the structure.

\textbf{summary():} For obtaining the summary statistics for the data
set using the summary() function, giving a brief rundown of the
numerical variables.

\textbf{library():} Used too load a specific package or library.

\begin{Shaded}
\begin{Highlighting}[]
\FunctionTok{head}\NormalTok{(cw\_data)}
\end{Highlighting}
\end{Shaded}

\begin{verbatim}
##   InvoiceNo StockCode                         Description Quantity
## 1    536365    85123A  WHITE HANGING HEART T-LIGHT HOLDER        6
## 2    536365     71053                 WHITE METAL LANTERN        6
## 3    536365    84406B      CREAM CUPID HEARTS COAT HANGER        8
## 4    536365    84029G KNITTED UNION FLAG HOT WATER BOTTLE        6
## 5    536365    84029E      RED WOOLLY HOTTIE WHITE HEART.        6
## 6    536365     22752        SET 7 BABUSHKA NESTING BOXES        2
##        InvoiceDate UnitPrice CustomerID        Country
## 1 01/12/2010 08:26      2.55      17850 United Kingdom
## 2 01/12/2010 08:26      3.39      17850 United Kingdom
## 3 01/12/2010 08:26      2.75      17850 United Kingdom
## 4 01/12/2010 08:26      3.39      17850 United Kingdom
## 5 01/12/2010 08:26      3.39      17850 United Kingdom
## 6 01/12/2010 08:26      7.65      17850 United Kingdom
\end{verbatim}

\begin{Shaded}
\begin{Highlighting}[]
\FunctionTok{summary}\NormalTok{(cw\_data)}
\end{Highlighting}
\end{Shaded}

\begin{verbatim}
##   InvoiceNo          StockCode         Description           Quantity        
##  Length:541909      Length:541909      Length:541909      Min.   :-80995.00  
##  Class :character   Class :character   Class :character   1st Qu.:     1.00  
##  Mode  :character   Mode  :character   Mode  :character   Median :     3.00  
##                                                           Mean   :     9.55  
##                                                           3rd Qu.:    10.00  
##                                                           Max.   : 80995.00  
##                                                                              
##  InvoiceDate          UnitPrice           CustomerID       Country         
##  Length:541909      Min.   :-11062.06   Min.   :12346    Length:541909     
##  Class :character   1st Qu.:     1.25   1st Qu.:13953    Class :character  
##  Mode  :character   Median :     2.08   Median :15152    Mode  :character  
##                     Mean   :     4.61   Mean   :15288                      
##                     3rd Qu.:     4.13   3rd Qu.:16791                      
##                     Max.   : 38970.00   Max.   :18287                      
##                                         NA's   :135080
\end{verbatim}

\begin{Shaded}
\begin{Highlighting}[]
\FunctionTok{library}\NormalTok{(dplyr) }\CommentTok{\#dplyr library is essentially a set of functions designed to make manipulating dataframes more intuitive and user{-}friendly.}
\end{Highlighting}
\end{Shaded}

\begin{verbatim}
## Warning: package 'dplyr' was built under R version 4.3.2
\end{verbatim}

\begin{verbatim}
## 
## Attaching package: 'dplyr'
\end{verbatim}

\begin{verbatim}
## The following objects are masked from 'package:stats':
## 
##     filter, lag
\end{verbatim}

\begin{verbatim}
## The following objects are masked from 'package:base':
## 
##     intersect, setdiff, setequal, union
\end{verbatim}

\hypertarget{data-diagnostic}{%
\subsection{Data Diagnostic}\label{data-diagnostic}}

The process of looking through data to find and fix problems,
irregularities, or trends that could affect its dependability and
quality is called \textbf{data diagnostic}. Making sure the data is
correct, comprehensive, and appropriate for analysis is part of it.

\textbf{(is.na(cw\_data)):} Used to calculate the number of missing
values in each column of the data set.

\textbf{print():} To display the missing values for each column to check
for any gaps in the data.

\begin{Shaded}
\begin{Highlighting}[]
\NormalTok{missing\_values\_per\_column }\OtherTok{\textless{}{-}} \FunctionTok{colSums}\NormalTok{(}\FunctionTok{is.na}\NormalTok{(cw\_data))}
\FunctionTok{cat}\NormalTok{(}\StringTok{"Missing Values Per Column:}\SpecialCharTok{\textbackslash{}n}\StringTok{"}\NormalTok{)}
\end{Highlighting}
\end{Shaded}

\begin{verbatim}
## Missing Values Per Column:
\end{verbatim}

\begin{Shaded}
\begin{Highlighting}[]
\FunctionTok{print}\NormalTok{(missing\_values\_per\_column)}
\end{Highlighting}
\end{Shaded}

\begin{verbatim}
##   InvoiceNo   StockCode Description    Quantity InvoiceDate   UnitPrice 
##           0           0           0           0           0           0 
##  CustomerID     Country 
##      135080           0
\end{verbatim}

The purpose of converting the \textbf{InvoiceDate} column to
\textbf{POSIXct} format is to improve its temporal evaluation
capabilities by making it more appropriate for time-based analysis.

After conversion, the \textbf{InvoiceDate} field undergoes a
comprehensive review to find and fix any errors. The objective is to
guarantee data accuracy.

\begin{Shaded}
\begin{Highlighting}[]
\FunctionTok{library}\NormalTok{(lubridate) }\CommentTok{\#lubridate package offers utilities that make date{-}time information manipulation, extraction, and computation easier. }
\end{Highlighting}
\end{Shaded}

\begin{verbatim}
## 
## Attaching package: 'lubridate'
\end{verbatim}

\begin{verbatim}
## The following objects are masked from 'package:base':
## 
##     date, intersect, setdiff, union
\end{verbatim}

\begin{Shaded}
\begin{Highlighting}[]
\NormalTok{cw\_data}\SpecialCharTok{$}\NormalTok{InvoiceDate }\OtherTok{\textless{}{-}} \FunctionTok{as.POSIXct}\NormalTok{(cw\_data}\SpecialCharTok{$}\NormalTok{InvoiceDate, }\AttributeTok{format =} \StringTok{"\%d/\%m/\%Y \%H:\%M"}\NormalTok{)}
\FunctionTok{sum}\NormalTok{(}\FunctionTok{is.na}\NormalTok{(cw\_data}\SpecialCharTok{$}\NormalTok{InvoiceDate))}
\end{Highlighting}
\end{Shaded}

\begin{verbatim}
## [1] 0
\end{verbatim}

\hypertarget{in-depth-exploratory-data-analysis}{%
\subsection{In-depth Exploratory Data
Analysis}\label{in-depth-exploratory-data-analysis}}

\hypertarget{question-1.-what-are-the-most-popular-products-and-how-do-their-sales-vary-over-time}{%
\subsubsection{Question 1. What are the most popular products and how do
their sales vary over
time?}\label{question-1.-what-are-the-most-popular-products-and-how-do-their-sales-vary-over-time}}

Analyzing the sale patterns and popularity of products is critical for
all businesses looking for strategic insights. With the goal of
understanding the most well-liked products and analyzing how there is a
fluctuation over time, the below analysis provides the useful data for
wise decision making in the ever evolving market environment.

\textbf{aggregate():} Used to calculate the summary statistics or
average of the given data set.

The products are then sorted in a descending order based on the quantity
to pinpoint the most popular product.

\begin{Shaded}
\begin{Highlighting}[]
\NormalTok{product\_sales }\OtherTok{\textless{}{-}} \FunctionTok{aggregate}\NormalTok{(cw\_data}\SpecialCharTok{$}\NormalTok{Quantity, }\AttributeTok{by =} \FunctionTok{list}\NormalTok{(}\AttributeTok{Product =}\NormalTok{ cw\_data}\SpecialCharTok{$}\NormalTok{Description), sum)}
\NormalTok{productlist\_sorted }\OtherTok{\textless{}{-}}\NormalTok{ product\_sales[}\FunctionTok{order}\NormalTok{(}\SpecialCharTok{{-}}\NormalTok{product\_sales}\SpecialCharTok{$}\NormalTok{x), ]}
\FunctionTok{head}\NormalTok{(productlist\_sorted,}\DecValTok{10}\NormalTok{)}
\end{Highlighting}
\end{Shaded}

\begin{verbatim}
##                                 Product     x
## 4118  WORLD WAR 2 GLIDERS ASSTD DESIGNS 53847
## 1916            JUMBO BAG RED RETROSPOT 47363
## 259       ASSORTED COLOUR BIRD ORNAMENT 36381
## 2805                     POPCORN HOLDER 36334
## 2460    PACK OF 72 RETROSPOT CAKE CASES 36039
## 4027 WHITE HANGING HEART T-LIGHT HOLDER 35317
## 2869                 RABBIT NIGHT LIGHT 30680
## 2219            MINI PAINT SET VINTAGE  26437
## 2426         PACK OF 12 LONDON TISSUES  26315
## 2458 PACK OF 60 PINK PAISLEY CAKE CASES 24753
\end{verbatim}

\begin{Shaded}
\begin{Highlighting}[]
\FunctionTok{cat}\NormalTok{(}\StringTok{"The Most Popular Product is"}\NormalTok{, productlist\_sorted}\SpecialCharTok{$}\NormalTok{Product[}\DecValTok{1}\NormalTok{],}\StringTok{"}\SpecialCharTok{\textbackslash{}n}\StringTok{"}\NormalTok{)}
\end{Highlighting}
\end{Shaded}

\begin{verbatim}
## The Most Popular Product is WORLD WAR 2 GLIDERS ASSTD DESIGNS
\end{verbatim}

The result shows that ``\textbf{WORLD WAR 2 GLIDERS ASSTD DESIGNS}'' is
the top-performing item.

\begin{Shaded}
\begin{Highlighting}[]
\FunctionTok{library}\NormalTok{(ggplot2)}
\end{Highlighting}
\end{Shaded}

\begin{verbatim}
## Warning: package 'ggplot2' was built under R version 4.3.2
\end{verbatim}

\begin{Shaded}
\begin{Highlighting}[]
\NormalTok{productlist\_top3 }\OtherTok{\textless{}{-}}\NormalTok{ cw\_data }\SpecialCharTok{\%\textgreater{}\%}
  \FunctionTok{filter}\NormalTok{(Description }\SpecialCharTok{\%in\%} \FunctionTok{c}\NormalTok{(}\StringTok{"WORLD WAR 2 GLIDERS ASSTD DESIGNS"}\NormalTok{, }\StringTok{"JUMBO BAG RED RETROSPOT"}\NormalTok{, }\StringTok{"ASSORTED COLOUR BIRD ORNAMENT"}\NormalTok{)) }\SpecialCharTok{\%\textgreater{}\%}
  \FunctionTok{mutate}\NormalTok{(}\AttributeTok{total\_sales =}\NormalTok{ Quantity }\SpecialCharTok{*}\NormalTok{ UnitPrice) }\CommentTok{\#the top 3 out of the top 5 products from the above list is added in this}

\NormalTok{productlist\_top3 }\OtherTok{\textless{}{-}}\NormalTok{ productlist\_top3 }\SpecialCharTok{\%\textgreater{}\%}
  \FunctionTok{filter}\NormalTok{(total\_sales }\SpecialCharTok{\textgreater{}=} \DecValTok{0}\NormalTok{)}
\FunctionTok{ggplot}\NormalTok{(productlist\_top3, }\FunctionTok{aes}\NormalTok{(}\AttributeTok{x =}\NormalTok{ InvoiceDate, }\AttributeTok{y =}\NormalTok{ Quantity, }\AttributeTok{color =}\NormalTok{ Description)) }\SpecialCharTok{+}
  \FunctionTok{geom\_point}\NormalTok{(}\AttributeTok{size =} \DecValTok{2}\NormalTok{, }\AttributeTok{alpha =} \FloatTok{0.8}\NormalTok{, }\AttributeTok{show.legend =} \ConstantTok{TRUE}\NormalTok{) }\SpecialCharTok{+}
  \FunctionTok{geom\_line}\NormalTok{(}\AttributeTok{size =} \FloatTok{1.5}\NormalTok{, }\AttributeTok{alpha =} \FloatTok{0.8}\NormalTok{, }\AttributeTok{linetype =} \StringTok{"solid"}\NormalTok{) }\SpecialCharTok{+}
  \FunctionTok{labs}\NormalTok{(}\AttributeTok{title =} \StringTok{"Top 3 Products: Sales Over Time"}\NormalTok{, }\AttributeTok{x =} \StringTok{"Date"}\NormalTok{, }\AttributeTok{y =} \StringTok{"Quantity"}\NormalTok{) }\SpecialCharTok{+}
  \FunctionTok{theme\_minimal}\NormalTok{() }\SpecialCharTok{+}
  \FunctionTok{scale\_color\_brewer}\NormalTok{(}\AttributeTok{palette =} \StringTok{"Set1"}\NormalTok{) }\SpecialCharTok{+}
  \FunctionTok{theme}\NormalTok{(}\AttributeTok{legend.position =} \StringTok{"top"}\NormalTok{, }
        \AttributeTok{legend.text =} \FunctionTok{element\_text}\NormalTok{(}\AttributeTok{size =} \DecValTok{7}\NormalTok{), }
        \AttributeTok{legend.title =} \FunctionTok{element\_blank}\NormalTok{(),}
        \AttributeTok{axis.text.x =} \FunctionTok{element\_text}\NormalTok{(}\AttributeTok{angle =} \DecValTok{45}\NormalTok{, }\AttributeTok{hjust =} \DecValTok{1}\NormalTok{)) }\SpecialCharTok{+}
  \FunctionTok{annotate}\NormalTok{(}\StringTok{"text"}\NormalTok{, }\AttributeTok{x =} \FunctionTok{max}\NormalTok{(productlist\_top3}\SpecialCharTok{$}\NormalTok{InvoiceDate), }
           \AttributeTok{y =} \FunctionTok{max}\NormalTok{(productlist\_top3}\SpecialCharTok{$}\NormalTok{total\_sales), }
           \AttributeTok{label =} \StringTok{""}\NormalTok{, }
           \AttributeTok{hjust =} \DecValTok{1}\NormalTok{, }\AttributeTok{vjust =} \DecValTok{1}\NormalTok{, }\AttributeTok{size =} \DecValTok{3}\NormalTok{, }\AttributeTok{color =} \StringTok{"gray50"}\NormalTok{) }\SpecialCharTok{+}
  \FunctionTok{scale\_x\_datetime}\NormalTok{(}\AttributeTok{date\_labels =} \StringTok{"\%b \%Y"}\NormalTok{, }\AttributeTok{date\_breaks =} \StringTok{"3 months"}\NormalTok{) }\SpecialCharTok{+}
  \FunctionTok{theme}\NormalTok{(}\AttributeTok{plot.title =} \FunctionTok{element\_text}\NormalTok{(}\AttributeTok{hjust =} \FloatTok{0.5}\NormalTok{, }\AttributeTok{size =} \DecValTok{18}\NormalTok{, }\AttributeTok{face =} \StringTok{"bold"}\NormalTok{),}
        \AttributeTok{plot.caption =} \FunctionTok{element\_text}\NormalTok{(}\AttributeTok{size =} \DecValTok{8}\NormalTok{, }\AttributeTok{color=}\StringTok{"gray50"}\NormalTok{))}
\end{Highlighting}
\end{Shaded}

\begin{verbatim}
## Warning: Using `size` aesthetic for lines was deprecated in ggplot2 3.4.0.
## i Please use `linewidth` instead.
## This warning is displayed once every 8 hours.
## Call `lifecycle::last_lifecycle_warnings()` to see where this warning was
## generated.
\end{verbatim}

\includegraphics{retail_assessment_guideline_files/figure-latex/unnamed-chunk-6-1.pdf}

Every product is represented by a colored line, with the quantity sold
shown on the y-axis and the dates (\textbf{InvoiceDate}) displayed on
the x-axis. You can see how the sales numbers of these particular
products have fluctuated over time with this visualization, which offers
insights into their popularity and possible seasonal fluctuations.

\ul{\textbf{Conclusion:}}

Thorough quality checks of the data set was carried out to find and fix
any missing values in the data set. The \textbf{InvoiceDate} information
is converted into an appropriate date-time format in order to enable
time-based analysis.

Based on total quantity sold, the most popular product was determined to
be ``\textbf{WORLD WAR 2 GLIDERS ASSTD DESIGNS}'' which offers
insightful information about customer preferences.

The top three best-selling products are ``\textbf{WORLD WAR 2 GLIDERS
ASSTD DESIGNS}'', ``\textbf{JUMBO BAG RED RETROSPOT}'' and
``\textbf{ASSORTED COLOUR BIRD ORNAMENT}''. The graph shows the sales
trends for these products over time. The graphic gives a brief summary
of how the sales quantities of each product have changed over time, with
each colored line representing a different product.

\hypertarget{question-2.-can-we-segment-customers-based-on-their-purchasing-behavior}{%
\subsubsection{Question 2. Can we segment customers based on their
purchasing
behavior?}\label{question-2.-can-we-segment-customers-based-on-their-purchasing-behavior}}

The objective of this question is to identify unique trends in consumer
purchasing behavior in order to inform customized marketing plans and
raise overall company performance.

\ul{\textbf{Data Filtering:}} Cleaning the dataset entails eliminating
any missing or erroneous customer entries in order to provide a refined
dataset that is called \textbf{filtered\_data}.

\begin{Shaded}
\begin{Highlighting}[]
\NormalTok{filtered\_data }\OtherTok{\textless{}{-}}\NormalTok{ cw\_data }\SpecialCharTok{\%\textgreater{}\%}
  \FunctionTok{filter}\NormalTok{(CustomerID }\SpecialCharTok{!=} \DecValTok{0}\NormalTok{) }\CommentTok{\#to filter out rows where CustomerID is equal to 0}
\FunctionTok{head}\NormalTok{(filtered\_data)}
\end{Highlighting}
\end{Shaded}

\begin{verbatim}
##   InvoiceNo StockCode                         Description Quantity
## 1    536365    85123A  WHITE HANGING HEART T-LIGHT HOLDER        6
## 2    536365     71053                 WHITE METAL LANTERN        6
## 3    536365    84406B      CREAM CUPID HEARTS COAT HANGER        8
## 4    536365    84029G KNITTED UNION FLAG HOT WATER BOTTLE        6
## 5    536365    84029E      RED WOOLLY HOTTIE WHITE HEART.        6
## 6    536365     22752        SET 7 BABUSHKA NESTING BOXES        2
##           InvoiceDate UnitPrice CustomerID        Country
## 1 2010-12-01 08:26:00      2.55      17850 United Kingdom
## 2 2010-12-01 08:26:00      3.39      17850 United Kingdom
## 3 2010-12-01 08:26:00      2.75      17850 United Kingdom
## 4 2010-12-01 08:26:00      3.39      17850 United Kingdom
## 5 2010-12-01 08:26:00      3.39      17850 United Kingdom
## 6 2010-12-01 08:26:00      7.65      17850 United Kingdom
\end{verbatim}

\begin{Shaded}
\begin{Highlighting}[]
\FunctionTok{nrow}\NormalTok{(filtered\_data) }\CommentTok{\#nrow() is used to check the total number of rows in the data set}
\end{Highlighting}
\end{Shaded}

\begin{verbatim}
## [1] 406829
\end{verbatim}

A new dataframe is created named \textbf{filtered\_data}, which contains
the records of all the valid customers.

\textbf{Summary:} The records with missing or invalid customer
identifiers have been eliminated.

\ul{\textbf{Strategic Customer Segmentation:}} Understanding consumer
behaviour through classifying them according to their buying activity is
the main goal of strategic segmentation.

\begin{Shaded}
\begin{Highlighting}[]
\NormalTok{segments }\OtherTok{\textless{}{-}}\NormalTok{ cw\_data }\SpecialCharTok{\%\textgreater{}\%}
  \FunctionTok{group\_by}\NormalTok{(CustomerID) }\SpecialCharTok{\%\textgreater{}\%}
  \FunctionTok{summarise}\NormalTok{(}
    \AttributeTok{TotalPurchases =} \FunctionTok{n}\NormalTok{(),}
    \AttributeTok{TotalSpending =} \FunctionTok{sum}\NormalTok{(Quantity }\SpecialCharTok{*}\NormalTok{ UnitPrice)}
\NormalTok{  )}
\FunctionTok{head}\NormalTok{(segments)}
\end{Highlighting}
\end{Shaded}

\begin{verbatim}
## # A tibble: 6 x 3
##   CustomerID TotalPurchases TotalSpending
##        <int>          <int>         <dbl>
## 1      12346              2            0 
## 2      12347            182         4310 
## 3      12348             31         1797.
## 4      12349             73         1758.
## 5      12350             17          334.
## 6      12352             95         1545.
\end{verbatim}

\begin{Shaded}
\begin{Highlighting}[]
\FunctionTok{nrow}\NormalTok{(segments)}
\end{Highlighting}
\end{Shaded}

\begin{verbatim}
## [1] 4373
\end{verbatim}

The \textbf{segments} dataframe is composed of metrics that are computed
for every customer, including \textbf{TotalPurchases} which is the total
amount obtained by multiplying \textbf{Quantity} by \textbf{UnitPrice,}
and \textbf{TotalSpending} which is the number of transactions.. The
data has been grouped together by \textbf{CustomerID} by the use of
dplyr package in R.

\textbf{Summary:} A complete summary dataframe, called \textbf{segments}
has been created that contains various customer segments and relevant
metrics like the number of purchases and total spending.

\ul{\textbf{Agglomerative Grouping Analysis:}} To identify groups within
the customers that correspond to their unique buying patterns.

For clustering, the segments of the dataframe are chosen so that the
relevant columns (\textbf{TotalPurchases} and \textbf{TotalSpending})
are selected. Three clusters \textbf{(k = 5)} have been created for the
customers based on their purchase metrics using the hierarchical
clustering (\textbf{hclust} function). In the \textbf{segments}
dataframe, the cluster labels are allocated to every customer.

\begin{Shaded}
\begin{Highlighting}[]
\NormalTok{clustering\_data }\OtherTok{\textless{}{-}}\NormalTok{ segments[, }\FunctionTok{c}\NormalTok{(}\StringTok{"TotalPurchases"}\NormalTok{, }\StringTok{"TotalSpending"}\NormalTok{)]}
\NormalTok{hc }\OtherTok{\textless{}{-}} \FunctionTok{hclust}\NormalTok{(}\FunctionTok{dist}\NormalTok{(clustering\_data))}
\NormalTok{clusters }\OtherTok{\textless{}{-}} \FunctionTok{cutree}\NormalTok{(hc, }\AttributeTok{k =} \DecValTok{5}\NormalTok{)}
\NormalTok{segments}\SpecialCharTok{$}\NormalTok{Cluster }\OtherTok{\textless{}{-}} \FunctionTok{as.factor}\NormalTok{(clusters)}
\FunctionTok{head}\NormalTok{(segments, }\DecValTok{10}\NormalTok{)}
\end{Highlighting}
\end{Shaded}

\begin{verbatim}
## # A tibble: 10 x 4
##    CustomerID TotalPurchases TotalSpending Cluster
##         <int>          <int>         <dbl> <fct>  
##  1      12346              2            0  1      
##  2      12347            182         4310  1      
##  3      12348             31         1797. 1      
##  4      12349             73         1758. 1      
##  5      12350             17          334. 1      
##  6      12352             95         1545. 1      
##  7      12353              4           89  1      
##  8      12354             58         1079. 1      
##  9      12355             13          459. 1      
## 10      12356             59         2811. 1
\end{verbatim}

\textbf{Summary:} A recently added factor variable named
``\textbf{Cluster}'' in the segments dataframe now denotes the assigned
cluster for each customer.

\ul{\textbf{Conclusion:}}

Through the removal of incomplete records, the systematic analysis
successfully segments customers based on their purchase behavior,
guaranteeing data accuracy. With the help of KPIs like
\textbf{TotalPurchases} and \textbf{TotalSpending}, the resulting
customer categories provide detailed information. By using clustering,
different buying patterns can be identified, allowing for more focused
marketing campaigns. These results offer practical information that
directs strategic choices and improves overall business success.

\hypertarget{question-3.-are-there-specific-countries-that-contribute-more-to-sales-or-have-unique-buying-patterns}{%
\subsubsection{Question 3. Are there specific countries that contribute
more to sales or have unique buying
patterns?}\label{question-3.-are-there-specific-countries-that-contribute-more-to-sales-or-have-unique-buying-patterns}}

By analyzing whether particular nations make a substantial contribution
to overall sales or display distinctive purchasing patterns, this
approach seeks to provide important insights into sales dynamics.
Businesses trying to customize strategy, distribute resources
effectively, and modify marketing tactics for a variety of markets must
comprehend these regional quirks.

The \textbf{summary()} function has been used to derive the summary
statistics of the data set.

\begin{Shaded}
\begin{Highlighting}[]
\FunctionTok{summary}\NormalTok{(cw\_data)}
\end{Highlighting}
\end{Shaded}

\begin{verbatim}
##   InvoiceNo          StockCode         Description           Quantity        
##  Length:541909      Length:541909      Length:541909      Min.   :-80995.00  
##  Class :character   Class :character   Class :character   1st Qu.:     1.00  
##  Mode  :character   Mode  :character   Mode  :character   Median :     3.00  
##                                                           Mean   :     9.55  
##                                                           3rd Qu.:    10.00  
##                                                           Max.   : 80995.00  
##                                                                              
##   InvoiceDate                       UnitPrice           CustomerID    
##  Min.   :2010-12-01 08:26:00.00   Min.   :-11062.06   Min.   :12346   
##  1st Qu.:2011-03-28 11:34:00.00   1st Qu.:     1.25   1st Qu.:13953   
##  Median :2011-07-19 17:17:00.00   Median :     2.08   Median :15152   
##  Mean   :2011-07-04 14:02:44.05   Mean   :     4.61   Mean   :15288   
##  3rd Qu.:2011-10-19 11:27:00.00   3rd Qu.:     4.13   3rd Qu.:16791   
##  Max.   :2011-12-09 12:50:00.00   Max.   : 38970.00   Max.   :18287   
##                                                       NA's   :135080  
##    Country         
##  Length:541909     
##  Class :character  
##  Mode  :character  
##                    
##                    
##                    
## 
\end{verbatim}

\begin{Shaded}
\begin{Highlighting}[]
\NormalTok{unique\_countries }\OtherTok{\textless{}{-}} \FunctionTok{unique}\NormalTok{(cw\_data}\SpecialCharTok{$}\NormalTok{Country)}
\FunctionTok{print}\NormalTok{(unique\_countries)}
\end{Highlighting}
\end{Shaded}

\begin{verbatim}
##  [1] "United Kingdom"       "France"               "Australia"           
##  [4] "Netherlands"          "Germany"              "Norway"              
##  [7] "EIRE"                 "Switzerland"          "Spain"               
## [10] "Poland"               "Portugal"             "Italy"               
## [13] "Belgium"              "Lithuania"            "Japan"               
## [16] "Iceland"              "Channel Islands"      "Denmark"             
## [19] "Cyprus"               "Sweden"               "Austria"             
## [22] "Israel"               "Finland"              "Bahrain"             
## [25] "Greece"               "Hong Kong"            "Singapore"           
## [28] "Lebanon"              "United Arab Emirates" "Saudi Arabia"        
## [31] "Czech Republic"       "Canada"               "Unspecified"         
## [34] "Brazil"               "USA"                  "European Community"  
## [37] "Malta"                "RSA"
\end{verbatim}

A list of all the \textbf{unique} countries in the data set are
displayed above.

Total sales for each nation are determined by multiplying the
\textbf{UnitPrice} by the \textbf{Quantity} as below.

\begin{Shaded}
\begin{Highlighting}[]
\NormalTok{sales\_by\_country }\OtherTok{\textless{}{-}}\NormalTok{ cw\_data }\SpecialCharTok{\%\textgreater{}\%}
  \FunctionTok{group\_by}\NormalTok{(Country) }\SpecialCharTok{\%\textgreater{}\%}
  \FunctionTok{summarise}\NormalTok{(}\AttributeTok{total\_sales =} \FunctionTok{sum}\NormalTok{(UnitPrice}\SpecialCharTok{*}\NormalTok{Quantity))}
\NormalTok{sales\_by\_country}
\end{Highlighting}
\end{Shaded}

\begin{verbatim}
## # A tibble: 38 x 2
##    Country         total_sales
##    <chr>                 <dbl>
##  1 Australia           137077.
##  2 Austria              10154.
##  3 Bahrain                548.
##  4 Belgium              40911.
##  5 Brazil                1144.
##  6 Canada                3666.
##  7 Channel Islands      20086.
##  8 Cyprus               12946.
##  9 Czech Republic         708.
## 10 Denmark              18768.
## # i 28 more rows
\end{verbatim}

The above helps in identifying the regions which contribute
significantly to the sales.

The distribution of total sales across several countries is calculated
below:

\begin{Shaded}
\begin{Highlighting}[]
\NormalTok{sales\_by\_country }\OtherTok{\textless{}{-}}\NormalTok{ sales\_by\_country }\SpecialCharTok{\%\textgreater{}\%}
  \FunctionTok{mutate}\NormalTok{(}\AttributeTok{Country =} \FunctionTok{reorder}\NormalTok{(Country, }\SpecialCharTok{{-}}\NormalTok{total\_sales))}

\FunctionTok{ggplot}\NormalTok{(sales\_by\_country, }\FunctionTok{aes}\NormalTok{(}\AttributeTok{x =}\NormalTok{ Country, }\AttributeTok{y =}\NormalTok{ total\_sales, }\AttributeTok{fill =}\NormalTok{ Country)) }\SpecialCharTok{+}
  \FunctionTok{geom\_bar}\NormalTok{(}\AttributeTok{stat =} \StringTok{"identity"}\NormalTok{, }\AttributeTok{color =} \StringTok{"black"}\NormalTok{) }\SpecialCharTok{+}
  \FunctionTok{labs}\NormalTok{(}\AttributeTok{title =} \StringTok{"Total Sales by Country"}\NormalTok{, }\AttributeTok{x =} \StringTok{"Country"}\NormalTok{, }\AttributeTok{y =} \StringTok{"Total Sales"}\NormalTok{) }\SpecialCharTok{+}
  \FunctionTok{theme\_minimal}\NormalTok{() }\SpecialCharTok{+}
  \FunctionTok{theme}\NormalTok{(}\AttributeTok{axis.text.x =} \FunctionTok{element\_text}\NormalTok{(}\AttributeTok{size=}\DecValTok{8}\NormalTok{,}\AttributeTok{angle =} \DecValTok{90}\NormalTok{, }\AttributeTok{hjust =} \DecValTok{1}\NormalTok{),}
        \AttributeTok{axis.title.x =} \FunctionTok{element\_text}\NormalTok{(}\AttributeTok{size =} \DecValTok{10}\NormalTok{),}
        \AttributeTok{plot.title =} \FunctionTok{element\_text}\NormalTok{(}\AttributeTok{size =} \DecValTok{18}\NormalTok{, }\AttributeTok{hjust=}\FloatTok{0.5}\NormalTok{, }\AttributeTok{face =} \StringTok{\textquotesingle{}bold\textquotesingle{}}\NormalTok{),}
        \AttributeTok{legend.text =} \FunctionTok{element\_text}\NormalTok{(}\AttributeTok{size =} \DecValTok{6}\NormalTok{),}
        \AttributeTok{legend.title =} \FunctionTok{element\_text}\NormalTok{(}\AttributeTok{size =} \DecValTok{6}\NormalTok{),}
        \AttributeTok{legend.position =} \StringTok{"top"}\NormalTok{,}
        \AttributeTok{legend.key.size =} \FunctionTok{unit}\NormalTok{(}\FloatTok{0.3}\NormalTok{, }\StringTok{"cm"}\NormalTok{))}\SpecialCharTok{+}
  \FunctionTok{scale\_y\_continuous}\NormalTok{(}\AttributeTok{labels =}\NormalTok{ scales}\SpecialCharTok{::}\FunctionTok{comma\_format}\NormalTok{(}\AttributeTok{scale =} \FloatTok{1e{-}6}\NormalTok{, }\AttributeTok{suffix =} \StringTok{"M"}\NormalTok{))}
\end{Highlighting}
\end{Shaded}

\includegraphics{retail_assessment_guideline_files/figure-latex/unnamed-chunk-12-1.pdf}

\begin{Shaded}
\begin{Highlighting}[]
\FunctionTok{ggplot}\NormalTok{(sales\_by\_country, }\FunctionTok{aes}\NormalTok{(}\AttributeTok{x =}\NormalTok{ Country, }\AttributeTok{y =}\NormalTok{ total\_sales, }\AttributeTok{fill =}\NormalTok{ Country)) }\SpecialCharTok{+}
  \FunctionTok{geom\_bar}\NormalTok{(}\AttributeTok{stat =} \StringTok{"identity"}\NormalTok{, }\AttributeTok{color =} \StringTok{"black"}\NormalTok{) }\SpecialCharTok{+}
  \FunctionTok{labs}\NormalTok{(}\AttributeTok{title =} \StringTok{"Total Sales by Country(Logarithmic Scale)"}\NormalTok{, }\AttributeTok{x =} \StringTok{"Country"}\NormalTok{, }\AttributeTok{y =} \StringTok{"Total Sales (log scale)"}\NormalTok{) }\SpecialCharTok{+}
  \FunctionTok{theme\_minimal}\NormalTok{() }\SpecialCharTok{+}
  \FunctionTok{theme}\NormalTok{(}\AttributeTok{axis.text.x =} \FunctionTok{element\_text}\NormalTok{(}\AttributeTok{size=}\DecValTok{8}\NormalTok{, }\AttributeTok{angle =} \DecValTok{90}\NormalTok{, }\AttributeTok{hjust =} \DecValTok{1}\NormalTok{),}
        \AttributeTok{axis.title.x =} \FunctionTok{element\_text}\NormalTok{(}\AttributeTok{size =} \DecValTok{10}\NormalTok{),}
        \AttributeTok{plot.title =} \FunctionTok{element\_text}\NormalTok{(}\AttributeTok{size =} \DecValTok{18}\NormalTok{, }\AttributeTok{hjust=}\FloatTok{0.5}\NormalTok{, }\AttributeTok{face =} \StringTok{\textquotesingle{}bold\textquotesingle{}}\NormalTok{),}
        \AttributeTok{legend.text =} \FunctionTok{element\_text}\NormalTok{(}\AttributeTok{size =} \DecValTok{6}\NormalTok{),}
        \AttributeTok{legend.title =} \FunctionTok{element\_text}\NormalTok{(}\AttributeTok{size =} \DecValTok{6}\NormalTok{),}
        \AttributeTok{legend.position =} \StringTok{"top"}\NormalTok{,}
        \AttributeTok{legend.key.size =} \FunctionTok{unit}\NormalTok{(}\FloatTok{0.3}\NormalTok{, }\StringTok{"cm"}\NormalTok{)) }\SpecialCharTok{+}
  \FunctionTok{scale\_y\_log10}\NormalTok{(}\AttributeTok{labels =}\NormalTok{ scales}\SpecialCharTok{::}\FunctionTok{comma\_format}\NormalTok{(}\AttributeTok{scale =} \FloatTok{1e{-}6}\NormalTok{, }\AttributeTok{suffix =} \StringTok{"M"}\NormalTok{))}
\end{Highlighting}
\end{Shaded}

\includegraphics{retail_assessment_guideline_files/figure-latex/unnamed-chunk-12-2.pdf}

For the purpose of visualizing total sales, the first code snippet
(scale\_y\_continuous) formats labels with commas and a ``M'' suffix on
a linear scale. A logarithmic scale is used in the second code snippet
(scale\_y\_log10) to provide a clearer depiction of the distribution of
the data, which is advantageous for data sets with a large range of
values.

The distribution of order quantity among different countries is shown
visually by a box plot below.

\begin{Shaded}
\begin{Highlighting}[]
\FunctionTok{ggplot}\NormalTok{(cw\_data, }\FunctionTok{aes}\NormalTok{(}\AttributeTok{x =} \FunctionTok{reorder}\NormalTok{(Country, Quantity, }\AttributeTok{FUN =}\NormalTok{ median), }\AttributeTok{y =}\NormalTok{ Quantity)) }\SpecialCharTok{+}
  \FunctionTok{geom\_boxplot}\NormalTok{( }\AttributeTok{color =} \StringTok{"royalblue"}\NormalTok{, }\AttributeTok{alpha =} \FloatTok{0.7}\NormalTok{) }\SpecialCharTok{+}
  \FunctionTok{labs}\NormalTok{(}\AttributeTok{title =} \StringTok{"Order Quantity Distribution by Country"}\NormalTok{, }\AttributeTok{x =} \StringTok{"Country"}\NormalTok{, }\AttributeTok{y =} \StringTok{"Order Quantity (in units)"}\NormalTok{) }\SpecialCharTok{+}
  \FunctionTok{theme\_minimal}\NormalTok{() }\SpecialCharTok{+}
  \FunctionTok{theme}\NormalTok{(}\AttributeTok{axis.text.x =} \FunctionTok{element\_text}\NormalTok{(}\AttributeTok{angle =} \DecValTok{45}\NormalTok{, }\AttributeTok{hjust =} \DecValTok{1}\NormalTok{),}
        \AttributeTok{plot.title =} \FunctionTok{element\_text}\NormalTok{(}\AttributeTok{size =} \DecValTok{18}\NormalTok{, }\AttributeTok{hjust=}\FloatTok{0.5}\NormalTok{, }\AttributeTok{face =} \StringTok{\textquotesingle{}bold\textquotesingle{}}\NormalTok{),}
        \AttributeTok{legend.position =} \StringTok{"none"}\NormalTok{)}
\end{Highlighting}
\end{Shaded}

\includegraphics{retail_assessment_guideline_files/figure-latex/unnamed-chunk-13-1.pdf}

The analysis of the box plot yields valuable information about the
central tendencies and dispersion of order quantity in various
countries. It also displays the distribution of order quantities among
several nations which highlights the variation in consumer behavior.

\textbf{Summary:} The data set was analysed to find important features,
highlighting major contributors by calculating total sales for each
distinct country and identifying unique countries. To clearly depict the
distribution of order quantities among nations as well as overall sales,
visualizations were used.

According to the data, the \textbf{UK accounts for a significant portion
of total sales}, highlighting the necessity of specialized tactics and
resource allocation. A significant concentration of overall sales in the
UK is shown by the data. This highlights the necessity of developing
strategies that are specific to the British market. Companies should use
their resources carefully in order to take advantage of UK consumers'
high spending power and unique purchasing habits.

Although the contributions from other nations might not be as great, the
understanding of unique purchasing habits offers useful direction for
modifying marketing strategies and formulating tactical choices.

\hypertarget{question-4-bonus.-how-do-cancellations-impact-overall-sales-trends}{%
\subsubsection{Question 4 (Bonus). How do cancellations impact overall
sales
trends?}\label{question-4-bonus.-how-do-cancellations-impact-overall-sales-trends}}

Analyzing how cancellations affect sales trends is essential to
understand the dynamics of operations. The goal of this study is to give
a thorough grasp of the complex relationship between sales and
cancellations to the company looking to enhance customer happiness and
optimize operations. This study directs strategic decisions for improved
corporate performance by finding trends and insights.

\ul{\textbf{Finding Cancelled Transactions:}} We filtered the data to
find transactions that had an invoice number beginning with
``\textbf{C}'' which indicates that the transaction was cancelled. The
outcome of this procedure was the establishment of a special dataframe
called ``\textbf{cancelled\_data},'' which contained only records of
cancelled transactions.

\begin{Shaded}
\begin{Highlighting}[]
\NormalTok{cancelled\_data }\OtherTok{\textless{}{-}}\NormalTok{ cw\_data }\SpecialCharTok{\%\textgreater{}\%}
  \FunctionTok{filter}\NormalTok{(}\FunctionTok{startsWith}\NormalTok{(InvoiceNo, }\StringTok{"C"}\NormalTok{))}

\NormalTok{sales\_plot }\OtherTok{\textless{}{-}}\NormalTok{ cw\_data }\SpecialCharTok{\%\textgreater{}\%}
  \FunctionTok{group\_by}\NormalTok{(InvoiceDate) }\SpecialCharTok{\%\textgreater{}\%}
  \FunctionTok{summarise}\NormalTok{(}
    \AttributeTok{TotalSales =} \FunctionTok{sum}\NormalTok{(Quantity }\SpecialCharTok{*}\NormalTok{ UnitPrice),}
    \AttributeTok{cancelledSales =} \FunctionTok{sum}\NormalTok{(}\FunctionTok{ifelse}\NormalTok{(}\FunctionTok{startsWith}\NormalTok{(InvoiceNo, }\StringTok{"C"}\NormalTok{), Quantity }\SpecialCharTok{*}\NormalTok{ UnitPrice, }\DecValTok{0}\NormalTok{))}
\NormalTok{  )}
\end{Highlighting}
\end{Shaded}

Grouping the original data by \textbf{InvoiceDate}, two key metrics were
calculated for each date:

\begin{itemize}
\item
  \ul{\textbf{TotalSales:}} The sum of \textbf{Quantity} multiplied by
  \textbf{UnitPrice} for all transactions.
\item
  \ul{\textbf{cancelledSales:}} The sum of Quantity multiplied by
  \textbf{UnitPrice} for cancelled transactions. This analysis generated
  a summary dataframe, ``\textbf{sales\_plot},'' capturing both total
  and cancelled sales trends over time.
\end{itemize}

\begin{Shaded}
\begin{Highlighting}[]
\FunctionTok{ggplot}\NormalTok{(sales\_plot, }\FunctionTok{aes}\NormalTok{(}\AttributeTok{x =}\NormalTok{ InvoiceDate)) }\SpecialCharTok{+}
  \FunctionTok{geom\_line}\NormalTok{(}\FunctionTok{aes}\NormalTok{(}\AttributeTok{y =}\NormalTok{ TotalSales, }\AttributeTok{color =} \StringTok{"Total Sales"}\NormalTok{), }\AttributeTok{size =} \FloatTok{1.2}\NormalTok{, }\AttributeTok{alpha =} \FloatTok{0.8}\NormalTok{) }\SpecialCharTok{+}
  \FunctionTok{geom\_line}\NormalTok{(}\FunctionTok{aes}\NormalTok{(}\AttributeTok{y =}\NormalTok{ cancelledSales, }\AttributeTok{color =} \StringTok{"cancelled Sales"}\NormalTok{), }\AttributeTok{size =} \FloatTok{1.2}\NormalTok{, }\AttributeTok{alpha =} \FloatTok{0.8}\NormalTok{, }\AttributeTok{linetype =} \StringTok{"dashed"}\NormalTok{) }\SpecialCharTok{+}
  \FunctionTok{labs}\NormalTok{(}\AttributeTok{title =} \StringTok{"Sales Dynamics: Total vs Cancelled"}\NormalTok{,}
       \AttributeTok{x =} \StringTok{"Date"}\NormalTok{,}
       \AttributeTok{y =} \StringTok{"Sales"}\NormalTok{,}
       \AttributeTok{color =} \StringTok{"Sales Type"}\NormalTok{) }\SpecialCharTok{+}
  \FunctionTok{scale\_y\_continuous}\NormalTok{(}\AttributeTok{labels =}\NormalTok{ scales}\SpecialCharTok{::}\FunctionTok{comma\_format}\NormalTok{()) }\SpecialCharTok{+}
  \FunctionTok{theme\_minimal}\NormalTok{() }\SpecialCharTok{+}
  \FunctionTok{theme}\NormalTok{(}\AttributeTok{legend.position =} \StringTok{"top"}\NormalTok{, }\AttributeTok{plot.title =} \FunctionTok{element\_text}\NormalTok{(}\AttributeTok{size =} \DecValTok{18}\NormalTok{, }\AttributeTok{hjust=}\FloatTok{0.5}\NormalTok{, }\AttributeTok{face =} \StringTok{\textquotesingle{}bold\textquotesingle{}}\NormalTok{),}
        \AttributeTok{legend.title =} \FunctionTok{element\_blank}\NormalTok{())}
\end{Highlighting}
\end{Shaded}

\includegraphics{retail_assessment_guideline_files/figure-latex/unnamed-chunk-15-1.pdf}

\ul{\textbf{Visualization of Total vs.~Cancelled Sales Trends:}} To show
how total and cancelled sales have changed over time, a dynamic line
plot is created above.

\textbf{Summary:}

\textbf{Analysis of Cancellations:} Segregated and recognized
transactions within the data set that were designated as cancellations.

\textbf{Sales Insights:} Provided an overview of overall sales trends
and cancellations within the data set by summarizing both total and
cancelled sales for each date.

\textbf{Visual Representation:} Using a line plot, it was possible to
clearly show how total and cancelled sales have changed over time.

Companies can use this research as a strategic compass to help them make
decisions that will improve customer experiences and increase
operational efficiency in the face of cancellations' negative effects on
overall sales trends.

\hypertarget{summary-key-takeaways}{%
\subsection{Summary \& Key Takeaways}\label{summary-key-takeaways}}

\ul{\textbf{Summary:}}

The ``\textbf{Online Retail.csv}'' data set was analysed using
structural exploration, data diagnostics, and strategic questions.

\textbf{Product Popularity:} Sales trends for the top three products
throughout time were visualized, and ``WORLD WAR 2 GLIDERS ASSTD
DESIGNS'' was found to be the best-performing product.

\textbf{Consumer Segmentation:} To gain insights into a range of
purchase behaviors, hierarchical clustering was employed for strategic
consumer segmentation.

\textbf{Sales by Country:} This analysis focused on the countries that
made up the majority of sales, with a particular emphasis on the
\textbf{UK}.

\textbf{Impact of Cancellations:} Provided operational insights by
examining the correlation between sales trends and cancellations.

\ul{\textbf{Key Takeaways:}}

\textbf{Product Strategy:} Based on the success of ``\textbf{WORLD WAR 2
GLIDERS ASSTD DESIGNS}'' adjust marketing plans and inventory.

\textbf{Customer-Centric Approach:} Use customer segmentation to improve
engagement and target marketing.

\textbf{Global Sales Tactics:} Take note of the \textbf{UK's}
significant contribution to overall sales and modify plans to account
for various national purchasing habits.

\textbf{Operational Optimization:} For increased customer happiness and
operational efficiency, comprehend the dynamics of cancellations.

For strategic decision-making, marketing optimization, and general
business success, this detailed analysis offers useful insights.

\textbf{Analysis by Critical Examination:}

Interesting patterns observed:

\begin{enumerate}
\def\labelenumi{\arabic{enumi}.}
\item
  \ul{Seasonal Trends:} Marketing and inventory planning depend heavily
  on seasonal sales patterns, which are influenced by occasions like
  festivals and holidays. Acknowledging peak seasons enables companies
  to coordinate plans, maximize stock, and satisfy increased customer
  demand at particular periods of the year.
\item
  \ul{Customer Behavior Anomalies:} Abnormalities in customer behavior
  are indicated by sudden swings in sales data that cannot be attributed
  to internal reasons. Customer behaviors might be impacted by outside
  factors like cultural or economic changes. Businesses can adjust their
  strategy to changing client preferences and external dynamics by
  recognising and comprehending these anomalies.
\item
  \ul{Product Combinations:} Consistent product matching analysis
  provides business insights. The ability to identify products that are
  regularly bought together allows for the development of packaged
  offers or tailored suggestions. This tactic improves the shopping
  encounter, promotes repeat business, and increases customer happiness.
\end{enumerate}

\end{document}
